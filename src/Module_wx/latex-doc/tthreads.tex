\section{Multithreading overview}\label{wxthreadoverview}

Classes: \helpref{wxThread}{wxthread}, \helpref{wxMutex}{wxmutex}, 
\helpref{wxCriticalSection}{wxcriticalsection}, 
\helpref{wxCondition}{wxcondition}

wxWidgets provides a complete set of classes encapsulating objects necessary in
multithreaded (MT) programs: the \helpref{thread}{wxthread} class itself and different
synchronization objects: \helpref{mutexes}{wxmutex} and 
\helpref{critical sections}{wxcriticalsection} with 
\helpref{conditions}{wxcondition}. The thread API in wxWidgets resembles to
POSIX1.c threads API (a.k.a. pthreads), although several functions have
different names and some features inspired by Win32 thread API are there as
well.

These classes will hopefully make writing MT programs easier and they also
provide some extra error checking (compared to the native (be it Win32 or Posix)
thread API), however it is still a non-trivial undertaking especially for large
projects. Before starting an MT application (or starting to add MT features to
an existing one) it is worth asking oneself if there is no easier and safer way
to implement the same functionality. Of course, in some situations threads
really make sense (classical example is a server application which launches a
new thread for each new client), but in others it might be a very poor choice
(example: launching a separate thread when doing a long computation to show a
progress dialog). Other implementation choices are available: for the progress
dialog example it is far better to do the calculations in the 
\helpref{idle handler}{wxidleevent} or even simply do everything at once
but call \helpref{wxWindow::Update()}{wxwindowupdate} periodically to update
the screen.

If you do decide to use threads in your application, it is strongly recommended
that no more than one thread calls GUI functions. The thread sample shows that
it {\it is} possible for many different threads to call GUI functions at once
(all the threads created in the sample access GUI), but it is a very poor design
choice for anything except an example. The design which uses one GUI thread and
several worker threads which communicate with the main one using events is much
more robust and will undoubtedly save you countless problems (example: under
Win32 a thread can only access GDI objects such as pens, brushes, \&c created by
itself and not by the other threads).

For communication between secondary threads and the main thread, you may use 
\helpref{wxEvtHandler::AddPendingEvent}{wxevthandleraddpendingevent}
or its short version \helpref{wxPostEvent}{wxpostevent}. These functions
have a thread-safe implementation so that they can be used as they are for
sending events from one thread to another. However there is no built in method
to send messages to the worker threads and you will need to use the available
synchronization classes to implement the solution which suits your needs
yourself. In particular, please note that it is \emph{not} enough to derive
your class from \helpref{wxThread}{wxthread} and 
\helpref{wxEvtHandler}{wxevthandler} to send messages to it: in fact, this does
\emph{not} work at all.


